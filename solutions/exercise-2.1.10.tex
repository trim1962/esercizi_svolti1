	\begin{align*}
	Debito=&Pagato+Sconto\\
	=&Pagato+\dfrac{Sp\cdot Debito}{100}\\
	Pagato=&Debito-\dfrac{Sp\cdot Debito}{100}\\
	=&Debito\left(1-\dfrac{Sp}{100}\right)\\
	Debito=&\dfrac{Pagato}{1-\dfrac{Sp}{100}}\\
	Debito=&\dfrac{\mbox{\EUR{171}}}{1-\dfrac{24}{100}}\\
	=&\mbox{\EUR{225}}
	\end{align*}
Si poteva, riflettendoci sopra, operare in questa maniera:
\[Debito=Pagato+Sconto\]
Quindi se lo sconto è il \SI{24}{\percent} del debito, il pagato è il \SI{76}{\percent}. Quindi
\begin{align*}
	P=&\dfrac{Pagato}{Debito}\cdot100\\
	76=&\dfrac{\mbox{\EUR{171}}}{Debito}\cdot 100\\
	Debito=&\dfrac{\mbox{\EUR{171}}}{76}\cdot 100\\
	=&\mbox{\EUR{225}}
\end{align*}
