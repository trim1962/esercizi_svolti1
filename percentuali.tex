\chapter{Percentuale, Interesse e sconto}
\section{Percentuale}
\tcbstartrecording
\begin{exercise}{Parte}
Calcola \SI{62}{\percent} di \num{707}
	\tcblower
\begin{align*}
	p={}&\dfrac{PA}{TU}\cdot 100\\
	PA={}&\dfrac{p. TU}{100}\\
={}&\dfrac{62\cdot 707}{100}\\
={}&\num{438.34}
\end{align*}
\end{exercise}
\begin{exercise}
	In un negozio il \SI{20}{\percent} dei clienti ha comprato un cellulare. Se in questo giorno il negozio ha avuto \num{745} clienti, quanti hanno acquistato un cellulare?
	\tcblower
	\begin{align*}
		p={}&\dfrac{PA}{TU}\cdot 100\\
		PA={}&\dfrac{p. TU}{100}\\
		={}&\dfrac{20\cdot 745}{100}\\
		={}&\num{149}
	\end{align*}
\end{exercise}
\begin{exercise}
	In una società di \num{300} soci il \SI{40}{\percent} sono insegnanti. Quanti sono gli altri soci?
	\tcblower
	\begin{align*}
		p={}&\dfrac{PA}{TU}\cdot 100\\
		PA={}&\dfrac{p. TU}{100}\\
		={}&\dfrac{40\cdot 300}{100}\\
		={}&\num{120}
		\intertext{Altri soci}
		={}&300-120\\
		={}&180
	\end{align*}
\end{exercise}
\begin{exercise}[no solution]
Un negozio ha avuto 860 visitatori. Se il \SI{70}{\percent} dei clienti ha comprato delle pere, quanti hanno acquistato delle pere
\end{exercise}
\begin{exercise}
Calcola quel numero il cui  \SI{40}{\percent} è \num{540}
	\tcblower
	\begin{align*}
p={}&\dfrac{PA}{TU}\cdot 100\\
40={}&\dfrac{540}{TU}\cdot 100\\
TU={}&\dfrac{540}{40}\cdot 100\\
={}&\num{1350 }
	\end{align*}
\end{exercise}
\section{Sconto}
\begin{exercise}
	Per l'aquisto di un melo è stato praticato uno sconto del  \SI{90}{\percent} pari a  \EUR{180}. Qual era il prezzo iniziale?
	\tcblower
	\begin{align*}
		p_s={}&\dfrac{SC}{PI}\cdot 100\\
		90={}&\dfrac{180}{PI}\cdot 100\\
		PI={}&\dfrac{180}{90}\cdot 100\\
		={}&\mbox{\EUR{200}}
	\end{align*}
\end{exercise}
\begin{exercise}
In un negozio mi hanno ridotto il prezzo del  \SI{24}{\percent} e ho risparmiato \EUR{36}. Quale era il prezzo iniziale e quanto ho speso?
	\tcblower
	\begin{align*}
		p_s={}&\dfrac{SC}{PI}\cdot 100\\
		24={}&\dfrac{36}{PI}\cdot 100\\
		PI={}&\dfrac{36}{24}\cdot 100\\
		={}&\mbox{\EUR{150}}\\
		PI={}&SC+PF\\
		\mbox{\EUR{150}}={}&PF+\mbox{\EUR{36}}\\
		PF={}&\mbox{\EUR{150}}-\mbox{\EUR{36}}\\
		={}&\mbox{\EUR{114}}\\
	\end{align*}
\end{exercise}
\begin{exercise}%[no solution]
	Un commerciante mette in vendita una coperta praticando il \SI{10}{\percent} di sconto. Dopo un mese  la offre  scontandola del \SI{30}{\percent}. Se lo sconto complessivo è stato di \EUR{300}, quale era il prezzo iniziale?
		\tcblower
	In questo esempio lo sconto viene applicato due volte. Per risolverlo bisogna tener presente che il secondo sconto viene calcolato non sul prezzo iniziale ma questo nuovo importo, il quale sarà uguale a quello fissato all'inizio meno il primo sconto.
	\begin{align*}
		ScontoFinale=&PrimoSconto+SecondoSconto=\mbox{\EUR{300}}\\
		\intertext{Definiamo delle abbreviazioni}
		Primoscontopercentuale=&PSP\\
		Secondoscontopercentuale=&SSP\\
		PrimoPrezzo=&PP\\
		SecondoPrezzo=&SP\\
	\end{alig*}
\begin{align*}
	PrimoSconto=&\dfrac{PSP\cdot PP}{100}\\
	SP=&PP-PrimoSsconto\\
	=&PP-\dfrac{PSP\cdot PP}{100}\\
	=&PP\left(1-\dfrac{PSP}{100}\right)\\
	SecondoSconto=&\dfrac{SSP\cdot SP}{100}\\&\dfrac{SSP\cdot SP}{100}\\
	
\end{align*}
\end{exercise}
\begin{exercise}
	Pagando prima un debito ho avuto uno sconto del  \SI{24}{\percent} e pago \EUR{171}. A quanto ammontava il debito?
	\tcblower
	\begin{align*}
	Debito={}&Pagato+Sconto\\
	={}&Pagato+\dfrac{Sp\cdot Debito}{100}\\
	Pagato={}&Debito-\dfrac{Sp\cdot Debito}{100}\\
	={}&Debito\left(1-\dfrac{Sp}{100}\right)\\
	Debito={}&\dfrac{Pagato}{1-\dfrac{Sp}{100}}\\
	Debito={}&\dfrac{\mbox{\EUR{171}}}{1-\dfrac{24}{100}}\\
	={}&\mbox{\EUR{225}}
	\end{align*}
Si poteva, riflettendoci sopra, operare in questa maniera:
\[Debito=Pagato+Sconto\]
Quindi se lo sconto è il \SI{24}{\percent} del debito, il pagato è il \SI{76}{\percent}. Quindi
\begin{align*}
	P={}&\dfrac{Pagato}{Debito}\cdot100\\
	76={}&\dfrac{\mbox{\EUR{171}}}{Debito}\cdot 100\\
	Debito={}&\dfrac{\mbox{\EUR{171}}}{76}\cdot 100\\
	={}&\mbox{\EUR{225}}
\end{align*} 
\end{exercise}
\begin{exercise}
Il prezzo del pane è di \SI{67}{\euro\per\kg}. Mi è stato venduto a \SI{43}{\euro\per\kg}. Che percentuale di sconto mi è stata praticata.
	\tcblower
	\begin{align*}
		Sconto={}&PI-PF\\
		P={}&\dfrac{Sconto}{PI}\cdot 100\\
		P={}&\dfrac{24}{67}\cdot 100\\
		={}&\SI{35.82}{\percent}
	\end{align*}
\end{exercise}

\section{Incrementi}
\begin{exercise}
	Un commerciante ha acquistato della carne a  \SI{42}{\euro\per\kg} e l'ha rivenduta a  \SI{52}{\euro\per\kg}. Che percentuale di guadagno sul prezzo iniziale.
	\tcblower
	\begin{align*}
		Incremento={}&PF-PI\\
		P={}&\dfrac{}{Incremento}\cdot 100\\
		P={}&\dfrac{10}{42}\cdot 100\\
		={}&\SI{23.81}{\percent}
	\end{align*}
\end{exercise}
\begin{exercise}
Il prezzo di un melo è di \EUR{37}. Viene rivenduto a \EUR{55}. Quanto è stato l'aumento percentuale?
	\tcblower
	\begin{align*}
		Incremento={}&PF-PI\\
		P={}&\dfrac{}{Incremento}\cdot 100\\
		P={}&\dfrac{18}{37}\cdot 100\\
		={}&\SI{48.65}{\percent}
	\end{align*}
\end{exercise} 
\begin{exercise}[no solution]
	Il costo di un bicchiere è di \EUR{201} ha avuto un incremento del \SI{20}{\percent}, quanto costa ora?
\end{exercise}
\begin{exercise}
	Un pollo dal costo di \EUR{47}  ha avuto un incremento del \SI{47}{\percent}. Quanto costa ora?
	\tcblower
	\begin{align*}
		p={}&\dfrac{INC}{PI}\cdot 100\\
		INC={}&\dfrac{p. PI}{100}\\
		={}&\dfrac{47\cdot 47}{100}\\
		={}&\mbox{\EUR{22.09}}\\ 
		PF={}&PI+INC\\
		={}&\mbox{\EUR{69.09}}
		\intertext{Altrimenti}
		PF={}&PI+INC\\
		PF={}&PI+\dfrac{p. PI}{100}\\
		={}&PI\left(1+\dfrac{p}{100}\right)\\
		={}&\num{47}\left(1+\dfrac{47}{100}\right)\\
		={}&\mbox{\EUR{69.09}}
	\end{align*}
\end{exercise}
\section{Interesse}
\begin{exercise}[no solution]
	Calcolare l'interesse semplice maturato su un capitale di \EUR{43586} impiegato al \SI{5}{\percent} per \num{7} anni.
 
\end{exercise} 
\begin{exercise}
	Calcolare l'interesse semplice maturato su un capitale di \EUR{1376} impiegato al \SI{13}{\percent} per \num{12} anni.
	\tcblower
	\begin{align*}
		I={}&\dfrac{C\cdot r}{100}\cdot t\\
		={}&\dfrac{1376\cdot 13}{100}\cdot 12\\
	={}&\mbox{\EUR{2146.56}}
	\end{align*}
\end{exercise} 
\begin{exercise}
	Calcolare l'interesse semplice maturato su un capitale di \EUR{41712} impiegato al \SI{18}{\percent} per \num{7} anni e \num{82} giorni.
	\tcblower
	\begin{align*}
		I={}&\dfrac{C\cdot r}{100}\cdot\left[a+ \dfrac{m}{12}+\dfrac{g}{360}\right]\\
		={}&\dfrac{41712\cdot 18}{100}\cdot\left[7+ \dfrac{0}{12}+\dfrac{82}{360}\right]\\
		={}&\mbox{\EUR{54267.31}}
	\end{align*}
\end{exercise} 
\begin{exercise}
	Calcolare l'interesse semplice maturato su un capitale di \EUR{75637} impiegato al \SI{14}{\percent} per \num{3} anni e \num{2} mesi.
	\tcblower
	\begin{align*}
		I={}&\dfrac{C\cdot r}{100}\cdot\left[a+ \dfrac{m}{12}+\dfrac{g}{360}\right]\\
		={}&\dfrac{75637\cdot 14}{100}\cdot\left[3+ \dfrac{2}{12}+\dfrac{0}{360}\right]\\
		={}&\mbox{\EUR{33532.40}}
	\end{align*}
\end{exercise} 
\begin{exercise}
	Calcolare l'interesse semplice maturato su un capitale di \EUR{9661} impiegato al \SI{9}{\percent} per \num{11} anni e \num{11} mesi e \num{147} giorni.
	\tcblower
	\begin{align*}
		I={}&\dfrac{C\cdot r}{100}\cdot\left[a+ \dfrac{m}{12}+\dfrac{g}{360}\right]\\
		={}&\dfrac{9661\cdot 9}{100}\cdot\left[11+ \dfrac{11}{12}+\dfrac{147}{360}\right]\\
		={}&\mbox{\EUR{10716.46}}
	\end{align*}
\end{exercise} 
\section{Montante}
\begin{exercise}
	Calcolare il montante, in regime di interesse semplice, maturato su un capitale di \EUR{83921} impiegato al \SI{10}{\percent} per \num{15} anni e \num{199} giorni.
	\tcblower
	\begin{align*}
		I={}&\dfrac{C\cdot r}{100}\cdot\left[a+ \dfrac{m}{12}+\dfrac{g}{360}\right]\\
		={}&\dfrac{83921\cdot 10}{100}\cdot\left[15+ \dfrac{0}{12}+\dfrac{199}{360}\right]\\
		={}&\mbox{\EUR{130520.47}}\\
		M={}&C+I\\
		={}&\mbox{\EUR{83921}}+\mbox{\EUR{130520.47}}\\
		={}&\mbox{\EUR{214441.47}}\\
	\end{align*}
\end{exercise}
\begin{exercise}
	Calcolare il montante, in regime di interesse semplice, maturato su un capitale di \EUR{17945} impiegato al \SI{6}{\percent} per \num{10} anni e \num{9} mesi.
	\tcblower
	\begin{align*}
		I={}&\dfrac{C\cdot r}{100}\cdot\left[a+ \dfrac{m}{12}+\dfrac{g}{360}\right]\\
		={}&\dfrac{17945\cdot 6}{100}\cdot\left[10+ \dfrac{9}{12}+\dfrac{0}{360}\right]\\
		={}&\mbox{\EUR{11574.53}}\\
		M={}&C+I\\
		={}&\mbox{\EUR{17945}}+\mbox{\EUR{11574.53}}\\
		={}&\mbox{\EUR{29519.53}}\\
	\end{align*}
\end{exercise}
\begin{exercise}
	Calcolare il montante, in regime di interesse semplice, maturato su un capitale di \EUR{51870} impiegato al \SI{1}{\percent} per \num{6} anni e \num{7} mesi e \num{66} giorni.
	\tcblower
	\begin{align*}
		I={}&\dfrac{C\cdot r}{100}\cdot\left[a+ \dfrac{m}{12}+\dfrac{g}{360}\right]\\
		={}&\dfrac{51870\cdot 1}{100}\cdot\left[6+ \dfrac{7}{12}+\dfrac{66}{360}\right]\\
		={}&\mbox{\EUR{3509.87}}\\
		M={}&C+I\\
		={}&\mbox{\EUR{51870}}+\mbox{\EUR{3509.87}}\\
		={}&\mbox{\EUR{55379.87}}\\
	\end{align*}
\end{exercise}
\tcbstoprecording
\newpage
\section{Soluzioni esercizi}
\tcbinputrecords