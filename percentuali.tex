\chapter{Percentuale, Interesse e sconto}
\section{Percentuale}
\tcbstartrecording
\begin{exercise}{Parte}
Calcola \SI{62}{\percent} di \num{707}
	\tcblower
\begin{align*}
	p=&\dfrac{PA}{TU}\cdot 100\\
	PA=&\dfrac{p. TU}{100}\\
=&\dfrac{62\cdot 707}{100}\\
=&\num{438.34}
\end{align*}
\end{exercise}
\begin{exercise}
	In un negozio il \SI{20}{\percent} dei clienti ha comprato un cellulare. Se in questo giorno il negozio ha avuto \num{745} clienti, quanti hanno acquistato un cellulare?
	\tcblower
	\begin{align*}
		p=&\dfrac{PA}{TU}\cdot 100\\
		PA=&\dfrac{p. TU}{100}\\
		=&\dfrac{20\cdot 745}{100}\\
		=&\num{149}
	\end{align*}
\end{exercise}
\begin{exercise}
	In una società di \num{300} soci il \SI{40}{\percent} sono insegnanti. Quanti sono gli altri soci?
	\tcblower
	\begin{align*}
		p=&\dfrac{PA}{TU}\cdot 100\\
		PA=&\dfrac{p. TU}{100}\\
		=&\dfrac{40\cdot 300}{100}\\
		=&\num{120}
		\intertext{Altri soci}
		=&300-120\\
		=&180
	\end{align*}
\end{exercise}
\begin{exercise}[no solution]
Un negozio ha avuto 860 clienti. Se il \SI{70}{\percent} dei clienti ha comprato delle pere, quanti hanno acquistato delle pere
\end{exercise}
\tcbstoprecording
\newpage
\section{Soluzioni esercizi}
\tcbinputrecords