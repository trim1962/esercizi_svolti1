\chapter{Numeri Naturali}
\label{cha:numerinaturali}
\section{\texorpdfstring{$\mcd$}{mcd} e \texorpdfstring{$\mcm$}{mcm}}
\label{sec:numnatmcmmcd}
\begin{esempiot} {mcm mcd}{}
Trovare il $\mcd$ e il $\mcm$  di \numlist{280;350;720}
\end{esempiot}
Iniziamo a scomporre i tre numeri
\stampapuntini
\begin{center}
	\begin{forest}
		[280
		[28
		[\puntini{7}]
		[4
		[2]
		[\puntini{2}]
		]
		]
		[10
		[5]
		[2]
		]
		]
	\end{forest}
	\begin{forest}
		[350
		[10
		[5]
		[2]
		]
		[\puntini{35}
		[\puntini{7}]
		[5]
		]
		]
	\end{forest}
	\begin{forest}
		[720
		[72
		[9
		[3]
		[\puntini{3}]
		]
		[\puntini{8}
		[4
		[2]
		[2]
		]
		[2]
		]
		]
		[\puntini{10}
		[5]
		[2]
		]
		]
	\end{forest}
\end{center}
Allineo i fattori

\begin{tabular}{lcllll}
	280 & = & \puntini{$2^3$} &  & 5  & 7 \\ 
	250 & = & 2     &  &\puntini{$5^2$}  & 7 \\ 
	720 & = & $2^4$ &\puntini{$3^2$}  & 5 &  \\  
\end{tabular} 

Ottengo

\begin{tabular}{lclcl}
	$\mcd$ & = & $2\cdot 5$ &=  & 10   \\ 
	$\mcm$ & = & $\puntini{2^4}\cdot 3^2\cdot5^2\cdot \puntini{7}$ &=  & 25200  \\ 
\end{tabular} 
\begin{esempiot}{}{}
	Trovare il $\mcd$ e il $\mcm$  di \numlist{432;270;288}
\end{esempiot}
Iniziamo a scomporre i tre numeri

\begin{center}
	\begin{forest}
	[432
	[\puntini{108}
	[2]
	[54
	[6
	[3]
	[2]]
	[\puntini{9}
	[3]
	[3]
	]
	]
	]
	[4
	[2]
	[2]
	]
	]
\end{forest}
\begin{forest}
[270
[\puntini{27}
[3]
[9
[3]
[3]
]
]
[\puntini{10}
[5]
[2]
]
]	
\end{forest}
\begin{forest}
	[288
	[4
	[2]
	[2]
	]
	[\puntini{72}
	[9
	[3]
	[3]
	]
	[\puntini{8}
	[4
	[2]
	[2]
	]
	[\puntini{2}]
	]
	]
	]
\end{forest}
	\end{center}
Allineo i fattori

\begin{tabular}{lclll}
432	&=  &$\puntini{2^4}$  & $3^2$ &  \\ 
270	& = & 2 & $\puntini{3^2}$ & 5 \\ 
288	& = &$2^5$  &$3^2$  &  \\ 
\end{tabular} 

Ottengo

\begin{tabular}{lclcl}
$\mcd$	&  =& $2\cdot 3^2$ &=  & 18 \\ 
$\mcm$	& = & $2^5\cdot \puntini{3^3}\cdot5$ &=  &\puntini{1440}  \\ 
\end{tabular} 
%\nonstampapuntini
\begin{esempiot}{}{}
	Trovare il $\mcd$ e il $\mcm$  di \numlist{70;48;78}
\end{esempiot}
Iniziamo a scomporre i tre numeri
\begin{center}
	\begin{forest}
		[70
		[7]
		[10
		[5]
		[2]
		]
		]
	\end{forest}
	\begin{forest}
		[48
		[6
		[3]
		[2]
		]
		[8
		[2]
		[4
		[2]
		[2]
		]
		]
		]
	\end{forest}
	\begin{forest}
		[84
		[2]
		[42
		[7]
		[6
		[3]
		[2]
		]
		]
		]
	\end{forest}
\end{center}
Allineo i fattori

\begin{tabular}{lcllll}
70	& = & 2     &  & 5 & 7 \\ 
48	& = & $2^4$ & 3 &   &   \\ 
84	& = & $2^2$ & 3 &   &  7\\ 
\end{tabular} 

Ottengo

\begin{tabular}{lclcl}
	$\mcd$	&  =& 2 &=  & 2 \\ 
	$\mcm$	& = & $2^4\cdot \puntini{3}\cdot5\cdot 7$ &=  &\puntini{1680}  \\
\end{tabular} 